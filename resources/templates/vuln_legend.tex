\begingroup
% --- lokalne style i makra tylko dla tej tabeli ---

% „Tag” dla poziomu: wymuś lokalne nadpisanie \sevpill
\makeatletter
\@ifundefined{sevpill}{%
  \newcommand{\sevpill}[2]{%
    \tikz[baseline=(P.base)]\node
      (P) [inner sep=1pt, outer sep=0pt, rounded corners=2pt,
           draw=#1!70!black, fill=#1!12, text=#1!40!black]
      {\footnotesize\bfseries\sffamily #2};%
  }%
}{%
  \renewcommand{\sevpill}[2]{%
    \tikz[baseline=(P.base)]\node
      (P) [inner sep=1pt, outer sep=0pt, rounded corners=2pt,
           draw=#1!70!black, fill=#1!12, text=#1!40!black]
      {\footnotesize\bfseries\sffamily #2};%
  }%
}
\makeatother

% Kolory pomocnicze dla tabeli (unikalne nazwy, bez kolizji)
\providecolor{VulnTblStripe}{HTML}{F5F5F5}
\providecolor{VulnTblRule}{HTML}{999999}

% Długości kolumn
\newlength{\VulnColA}
\newlength{\VulnColC}
\setlength{\VulnColA}{28mm}
\setlength{\VulnColC}{20mm}

% Pomocnicze makra do układu
\newcommand{\pillscale}{1.0}
\newcommand{\scalepill}[2]{%
  \raisebox{0pt}[0pt][0pt]{\scalebox{\pillscale}{\sevpill{#1}{#2}}}%
}
\newcommand{\pillbox}[2]{\makebox[\VulnColA][c]{\scalepill{#1}{#2}}}
\newcommand{\cvss}[1]{\textcolor{gray!50!black}{\texttt{#1}}}

% Ustawienia tabeli
\arrayrulecolor{VulnTblRule}
\setlength{\tabcolsep}{6pt}
\renewcommand{\arraystretch}{1.3}

\begin{table}[H]
\centering
\rowcolors{2}{VulnTblStripe}{white}
\begin{tabularx}{\linewidth}{@{} >{\centering\arraybackslash}p{\VulnColA} >{\arraybackslash}X >{\centering\arraybackslash}p{\VulnColC} @{}}
\rowcolor{gray!20}
\textbf{Poziom} & \textbf{Kryteria skrótowe} & \textbf{CVSS (v4.0)} \\
\hline
\pillbox{Crit}{KRYTYCZNY} &
Podatności o najwyższym stopniu zagrożenia, które pozwalają na \textbf{zdalne wykonanie kodu} lub \textbf{całkowite przejęcie kontroli nad systemem} \emph{bez dodatkowych warunków}. Ich wykorzystanie prowadzi do \textbf{pełnej utraty poufności, integralności i dostępności}. Wymagają \textbf{natychmiastowej reakcji} i \textbf{pilnego wdrożenia poprawek}. &
\cvss{9.0-10.0} \\
\pillbox{High}{WYSOKI} &
Podatności, które stwarzają \textbf{poważne ryzyko} i mogą prowadzić do \textbf{eskalacji uprawnień}, uzyskania \textbf{nieautoryzowanego dostępu} do danych poufnych lub \textbf{zakłócenia działania systemu}. Ich wykorzystanie często wymaga \emph{specyficznych warunków}, jednak potencjalne szkody są znaczne. Powinny być traktowane jako \textbf{wysoki priorytet} w procesie zarządzania ryzykiem. &
\cvss{7.0-8.9} \\
\pillbox{Med}{ŚREDNI} &
Podatności o \textbf{ograniczonym wpływie} na bezpieczeństwo, które mogą być wykorzystane jedynie w \emph{określonych warunkach} lub wymagają \emph{interakcji użytkownika} (np. kliknięcia w złośliwy link). Ich eliminacja może być realizowana w ramach \textbf{planowych działań utrzymaniowych}, jednak nie należy ich ignorować. &
\cvss{4.0-6.9} \\
\pillbox{Low}{NISKI} &
Podatności o \textbf{marginalnym wpływie} na bezpieczeństwo środowiska. Ich wykorzystanie wymaga \emph{bardzo specyficznych warunków technicznych} lub \emph{lokalnego dostępu} do systemu. W większości przypadków nie prowadzą do poważnych szkód. Ich usunięcie może być realizowane \textbf{w dalszej kolejności}, po wyeliminowaniu podatności o wyższym priorytecie. &
\cvss{0.1-3.9} \\
\pillbox{Info}{INFORMACYJNY} &
Zjawiska, które \textbf{nie stanowią bezpośredniego zagrożenia}, lecz mogą wskazywać na \textbf{błędy w konfiguracji} lub inne problemy. Chociaż \emph{nie wymagają natychmiastowych działań}, ich naprawa jest zalecana w celu \textbf{poprawy ogólnej higieny cyberbezpieczeństwa}. &
\cvss{—} \\
\hline
\end{tabularx}
\end{table}
\endgroup